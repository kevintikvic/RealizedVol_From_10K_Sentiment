\section{Data and Sample Description}
\label{sec: data_sample}
\nomenclature{CIK}{Central Index Key}

This section will initially outline the data collection and sample creation process. Then it will provide insights regarding the composition of the final corpus containing 46,483 annual filings as well as descriptive statistics about those filings. Finally, general textual features about the 10-K* corpus as well as descriptive statistics about the variables used, will be depicted. 

% ----------------------------------------------------------- %

\subsection{Data Collection, Sample Formation and Matching}
\label{ssec: data_sample_collection-matching}

The aim of this section is to outline the data collection and sampling procedure. Firstly, all 10-K* filings filed with the SEC between 1994 and 2017 were downloaded from \url{https://sraf.nd.edu/}, a software repository maintained by \citeauthor{Loughran2014} at the University of Notre Dame, which was developed during their works on 10-K* filings \parencite{Loughran2011, Loughran2014}. They provide a collection of what they refer to as \enquote{stage one parse files}, i.e., filings that were already parsed and cleaned from mark-up language tags such as HTML or XBRL as well as \enquote{non-textual} information embedded in the filings (e.g., encoded graphics, tables, files, pdf-documents, etc.)\footnote{\label{fn: parsing_LM}All parsing details are provided at \url{https://sraf.nd.edu/data/stage-one-10-x-parse-data/}. One important note regarding tables in the filings: some companies (for formatting purposes) apply tables to demarcate separate sections of the filing. Thus, by simply stripping out all text that occurs between mark-up tags \texttt{<TABLE>} and \texttt{</TABLE>}, one might actually lose relevant paragraphs containing textual contents. Thus, \citeauthor{Loughran2014} only eliminate those tables where the ratio of numeric characters to total (i.e., numeric plus alphabetic) characters is larger than ten percent.}. Thereby, \citeauthor{Loughran2014} make available a much more structured corpus of 10-K* filings, as they ease the researcher's burden of dealing with \enquote{raw} downloads from the SEC EDGAR web-page.

The initial, full sample contains 295,746 10-K* filings, submitted by 40,264 firms in the time period between 01/01/1994 and 12/31/2017. However, due to the necessity of collecting time series of prices for volatility calculation as well as other control variables (as pointed out in the description of the research design in sections \ref{sec: sentiment_calcs} and \ref{sec: volamodel}), the sample of usable 10-K*s reduces in the data matching procedure. Table \ref{tab: sample-cleaning} describes the sample construction process in greater detail. As one can see, for roughly sixty percent of observations there are stock tickers available, based on which the price series can be downloaded\footnote{The matching of CIK (\textit{central index key}), the identifier used by the SEC, and stock tickers is based on the following list as of 06/11/2018: \url{http://rankandfiled.com/\#/data/tickers}.}. The time series of stock prices and trading volumes were obtained from the Yahoo! Finance database in the time period between 06/05/2018 and 06/10/2018 and was available for roughly 42 percent of the identified tickers. In addition, the sample diminishes further when matching the 10K* corpus with data for the corresponding control variables, which were collected from the Compustat database, available via WRDS\footnote{\url{https://wrds-web.wharton.upenn.edu/wrds/}, accessed on 06/13/2018.}. The final sample of 10-K*s after matching the filings with Yahoo! Finance prices and Compustat WRDS data contains 46,483 filings from 3,736 companies for the time period from 01/06/1999 till 12/29/2017. 

\subsection{Sample Composition: Filing Types and Timing}
\label{ssec: data_sample_timingtypes}

Some distributional properties of the 46,483 available filings are displayed in Figure \ref{fig: no_reports_ymd_typestacked}, where the bars represent the number of filings in the respective year, month, day of the month, or weekday; the coloured stacks \textit{within} bars indicate the filing type\footnote{The stacked bars are hardly distinguishable due to the fact that the large majority of reports are of \enquote{standard} type 10-K, as indicated by the darkest-blue stacks, which dominate across years, months as well as days of month and week. In fact, 42,762 of the 46,483 reports (i.e., 91.99 percent) are either \enquote{vanilla} 10-K or amendments thereof (10-K-A). The detailed decomposition of filing types is provided in Table \ref{tab: type-composition}, and potential differences in post-filing volatility that are attributable to different filing types will be accounted for by adding filing type dummy variables in the regression tests.}. As one can observe in the upper-left panel (A), the number of filings follows an upward trend over the inspected time period, with the count of 10-K* submissions increasing by about 47 percent (comparing 2017 with 1999). On average, there are 2,446 filings available for each year in the sample (median: 2,515). Further, the upper-right panel (B) indicates that the vast majority (namely more than 41 percent) of 10-K*s are submitted in March. This is mainly due to the combination of most companies having a fiscal year deadline that coincides with the calendar year and the requirement of the SEC to submit the filing within 60-90 days after fiscal year end\footnote{See \url{https://www.sec.gov/fast-answers/answers-form10khtm.html} for details regarding filing deadlines.}. The lower-left graph (panel (C)) displays the distribution over days of the month; the two humps give rise to the conjecture that firms with fiscal years ending on either the 15th or the 30th (31st) day of the month will tend to submit filings just a few days before (or on) the deadline day. The lower-right panel (D) in Figure \ref{fig: no_reports_ymd_typestacked} shows the distribution of filings across weekdays, with Fridays leading ahead as submission weekday\footnote{Note: No 10-K* were filed on Saturdays or Sundays.}. These observations provide support for the choice to include year and month dummy variables to control for trends and seasonality as well as weekday- and day of month indicators to account for potential \enquote{weekend} or \enquote{turn of month} effects (or, phrased more generally, to control for so-called \enquote{calendar effects} in the filing behaviour). 

\subsection{Descriptive Statistics About the Corpus}
\label{ssec: data_sample_desc-stats-corpus}

This subsection provides descriptive statistics about the textual aspects of the 10-K* corpus used in this thesis. The corpus composition over years is displayed in Table \ref{tab: corpus-composition-p.a.}; in addition, the total, mean, and median number for both tokens and types (i.e., unique tokens)  is presented. As can be observed, both average and median word count increased in the past 19 years -- as was already evidenced in \textcite{Kogan2009_1} and \textcite{Rekabsaz2017}. The former especially investigated upon the jump in word counts occurring in 2002-03: this increase  in verboseness can potentially be attributed to the enactment of the Sarbanes-Oxley Act in 2002.

Moreover, Table \ref{tab: top5_each_LM}, presents the top-5 words for each LM-lexicon over the whole sample period (1999-2017) as well as three sub-groups. It appears that this tail of the distribution of word counts is stable over time, as most of the five top-words appear in each of the three sub-groups (very often even in the same ranking). This gives rise to optimism when applying weights that are based on past term frequencies to out-of-sample count observations.

\subsection{Descriptive Statistics About the Variables}
\label{ssec: data_sample_desc-stats-variables}
Concluding this section, some descriptive statistics about the variables of the linear model described in equation \eqref{eq: linear_reg} will be displayed and commented. Table \ref{tab: descr_stats_allvars} provides standard distributional figures such as the number of observations, the sample mean, sample standard deviation, 25/50/75 percentiles, as well as minimum and maximum value observed in the sample.

One can see that \texttt{PFRV} (expressed in logarithmic form) is ranging from .015 percent to 1,091 percent. As the latter clearly appears to be an outlier, the average realized volatility in the full sample is in range on a weekly level and amounts to 2.19 percent, whereas the median is slightly lower (2.11 percent). For the pre-filing counterpart, the figures are highly similar, yet a little lower, thereby indicating that \texttt{PreFRV} \textbf{might} -- on a univariate basis -- be a too optimistic predictor for \texttt{PFRV} by producing too small post-filing volatility estimates. 
%Considerably larger figures can be found for the sample statistics of the two time-series models (GARCH and GJR-GARCH) with average (median) values of 5.29 (5.03) and 4.98 (4.78) percent, respectively; thereby potentially resembling the empirical fact that GARCH-family models tend to be overly cautious, i.e., overestimate volatility very often. 
Further details on the distribution of \texttt{PFRV} and \texttt{PreFRV} as well as their univariate relationship will be provided in the section on univariate results (cf. section \ref{sssec: results_correls_mz}). 

With respect to the subsequent rows of Table \ref{tab: descr_stats_allvars}, which are related to the textual content of the 10-K*, one needs to highlight that variables \texttt{NEG\_SENT}, \texttt{POS\_SENT}, \texttt{ASSERT}, \texttt{UNCERT}, and \texttt{LITI} are \texttt{VIBTW} scores based on estimated weights that stem from the \textbf{full} sample (no in- versus out-of-sample split was applied in computing the weights and scores). Note that sentiment \textit{scores} are displayed rather than weights, as the latter, due to the z-standardization described in equation \eqref{eq: wv_estimates}, have zero sample mean and unit variance (standard deviation) by construction. As sentiment scores per se are difficult to interpret in terms of their scale, Figure \ref{fig: histos_sentiscores} exhibits six histograms for the five sentiment scores as well as the readability measure for illustrative purposes and ease of interpretation. The distribution of scores appears close to Gaussian and is certainly symmetric, while readability appears to be slightly more skewed (to the left) as well as mesokurtic (compared to the normal distribution). The last text-related variable captures the focus on financial topics (\texttt{FIN}); considering that it is calculated as the logarithm of one plus the term count of financial keywords, one can see that the term count ranges from zero up to 3,363, with the mean frequency being equal to 493 financial expressions per filing.

The remaining variables serve as control variables to account for a potential confounding effect; for instance, smaller and higher levered firms have been shown to be more risky. In terms of size (measured as logarithm of the firm's total assets), the sample ranges from small firms with an asset base of three million USD on the lower end (ignoring one percent of outliers) to large financial institutions like Citigroup or JPMorgan Chase with assets of multiple trillion USD. The latter represents the maximum in the sample with the 10-K* filed in 2017 and total assets of 2.49 trillion USD. The average (median) company in the sample operates with assets of 670 (600) million USD. With respect to book-to-market-ratio (\texttt{BTM}), the distribution appears largely driven by outliers (minimum of zero and maximum of 4,040 for a mean (median) of .75 (.49)). So as to dampen this distributional distortions, this variable is used in logarithmic form, making it more Gaussian yet leaving visible leptokurtosis. A similar picture is obtained for variable \texttt{TRVOL}, measuring the median trading volume in the week preceding the 10-K* filing. The measure ranges for a single share to 511 million shares traded, with the average being 1.23 million shares traded a day. Applying the logarithmic transformation helps to standardize the variable, making it close to normal (skewness: -.4, kurtosis: 2.89). Control variable \texttt{VIX} serves to account for market-level volatility, and it should be highlighted that the descriptive statistics in Table \ref{tab: descr_stats_allvars} do not apply for the VIX during the whole sample period (1999-2017). Instead, the variable represents the median value of the VIX in the week preceding the filing date. As indicated, \texttt{VIX} on average had a value of 19.68 (median slightly lower with 17.86), reaching its minimum (9.43) during the very calm market period in July 2017 and its maximum (72.67) in late November 2008 as a consequence of the crash in the housing market and the subsequent start of the financial crisis. Finally, the filing company's leverage ratio is considered as control variable. Among the 46,483 corporate filings available for analysis, \texttt{LEVER} on average was 52 percent, and was within the economic and accounting \enquote{boundaries} of zero (full-equity) and unity (full-debt). 

\clearpage